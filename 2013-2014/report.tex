\documentclass[12pt]{report}
\usepackage[utf8]{inputenc}
\usepackage[french]{babel}
\usepackage{graphicx}
\usepackage{titling}
\usepackage[margin=1in]{geometry}
\setlength{\droptitle}{-6.133742cm}
\title{Rapport d'activité de l'association GConfs pour l'année 2013 - 2014}
\author{Paul `Dettorer' Hervot (hervot\_p)
        \and Alexis `Horgix' Chotard (chotar\_a)
        \and Kaci `Bruce' Adjou (adjou\_k)
        \and Antoine `Zeletochoy' Lecubin (lecubi\_a)}

\begin{document}
  \begin{titlepage}
    \begin{center}
      \includegraphics[scale=0.4]{logo_gconfs.png} \\
    \end{center}
    {\let\newpage\relax\maketitle}
  \end{titlepage}
  \chapter{Premier Semestre}
  \section{Objectifs du Bureau pour l'année 2013-2014}
  \begin{itemize}
    \item
      Décentraliser les décisions afin que l'association ne dépende pas de son
      bureau\\ L'ancien président, Kévin Sztern, a fait part lors de l'assemblée
      générale précédente de sa décéption vis à vis de la dépendence dont fait
      preuve l'association envers son Bureau, nous avons donc pour objectif de
      réduire cette dépendence aux tâches critiques (budget, documents
      administratifs).
    \item
      Mise en place d'un système de responsables pour les conférences\\ Cet
      objectif découle du précédent : le responsable aura pour tâche de réserver
      la salle de conférence et les différentes salles de TP, de s'assurer que
      la conférence sera enregistrée.  Ceci permettra alors de libérer le bureau
      de ces différentes tâches.
    \item
      Mise en place d'un enregistrement et d'un suivi en direct des
      conférences\\ Lors de l'année précédente nous avons expérimenté
      l'enregistrement et la retransmission live par le biais de Mathieu
      TRENTIN. Celle-ci ne nous convenait pas sur plusieurs points, dont deux
      princpaux :
      \begin{itemize}
        \item
          Le temps \\ Il nous était demandé de réserver ce service au maximum
          deux semaines à l'événement, or il ne nous est pas possible de
          planifier des conférences plus d'un mois à l'avance.
        \item
          Les enregistrements \\ Il arrivait à Mathieu TRENTIN de perdre les
          enregistrements et ce sans aucune possibilité de les récupérer autre
          part.
      \end{itemize}
      Il a donc été décidé d'investir dans du matériel audio-visuel afin de
      pouvoir assurer l'enregistrement et la retransmission systématique de nos
      conférences.
    \item
      Invitation de conférenciers externes\\ Jusqu'alors, les conférences
      organisées par l'assocation étaient présentées par des membres de
      l'association ou des élèves de l'EPITA, afin de diversifier nos sujets
      nous avons décidé de faire venir des conférenciers externes à GConfs et au
      groupe IONIS.
    \item Poursuivre les partenariats conclus en 2012 - 2013
      \begin{itemize}
        \item Les partenariats ci-dessous sont destinés à être poursuivis par le bureau :
        \begin{itemize}
          \item Assistants
          \item VJN
        \end{itemize}
      \item Les partenariats suivants sont à l'étude
        \begin{itemize}
          \item Evolutek
          \item LaCity
        \end{itemize}
      \end{itemize}
  \end{itemize}
  \section{Budget prévisionnel}
  Un million de dollars.
  \section{Conférences réalisées durant ce semestre}
  \begin{itemize}
    \item Retour d'expérience d'un Stagiaire à Google par Pierre 'delroth' Bourdon (bourdo\_p)
    \item High Frequency Trading par Julien 'snooze92' Lehuen
    \item Introduction à OCAML par Philémon 'philgekni' Gardet (gardet\_p)
    \item UNIX, make it simple par Paul 'Dettorer' Hervot (hervot\_p), Clément 'wxcafe' Hertling (hertli\_c) et Kaci 'Bruce' Adjou (adjou\_k)
    \item Introduction au C\# par Raphaël 'shugo' Boissel (boisse\_r)
    \item Ocaml : Under the Hood par Théophile 'yroeht' Ranquet (ranque\_t)
    \item Bien démarrer son Projet par Valentin 'toogy' Iovene (iovene\_v), Alexis 'Horgix' Chotard (chotar\_a) et Théophile 'yroeht' Ranquet (ranque\_t)
    \item Pixel Art par Thomas 'bonhomme' Chassin (chassi\_t) et Pierre Guyot
    \item Graphics par Raphaël 'shugo' Boissel (boisse\_r)
    \item Graphics avancée par Raphaël 'shugo' Boissel (boisse\_r)
    \item 22 conférences lors de la Nuit de l'info, par : \\
      \begin{itemize}
        \item Pierre 'ptishell' Surply (surply\_p)
        \item Raphaël 'shugo' Boissel (boisse\_r)
        \item Paul 'Dettorer' Hervot (hervot\_p)
        \item Mickaël 'mogmi' Bidon (bidon\_m)
        \item Rémi 'halfr' Audebert (audber\_a)
        \item Julien Déléan
        \item Kévin 'Chewie' Sztern (sztern\_k)
        \item Thomas 'Mackwic' Wickham
        \item Frédéric 'skikepok' Lefort (lefort\_f)
        \item Jean-Luc 'thiel' Bounthong (bounth\_j)
        \item Antoine 'Serialk' Pietri (pietri\_a)
        \item Valentin 'nitnelave' Tolmer (tolmer\_v)
        \item Antoine 'Zeletochoy' Lecubin (lecubi\_a)
        \item Adrien 'schischi' Schildknecht (schild\_a)
      \end{itemize}
    \item Moteurs Physiques par Sébastien Crozet (crozet\_s)
    \item GNU Bison par Valentin 'nitnelave' Tolmer (tolmer\_v) et Théophile 'yroeht' Ranquet (ranque\_t)
  \end{itemize}
  \section{Notation des membres}
      TODO
  \section{Rapports des membres}
      TODO
  \chapter{Second Semestre}
  \section{Avancement des objectifs}
  \section{Budget}
  \section{Notation des Membres}
  \section{Rapports des membres}
\end{document}

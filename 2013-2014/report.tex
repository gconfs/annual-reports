\documentclass[12pt]{report}
\usepackage[utf8]{inputenc}
\usepackage[french]{babel}
\usepackage{graphicx}
\usepackage{titling}
\usepackage[margin=1in]{geometry}
\usepackage[toc,titletoc]{appendix}
\usepackage{pdfpages}
\usepackage{eurosym}
\setlength{\droptitle}{-6.133742cm}
\title{Rapport d'activité de l'association GConfs pour l'année 2013 - 2014}
\author{Paul `Dettorer' Hervot (hervot\_p)
        \and Alexis `Horgix' Chotard (chotar\_a)
        \and Kaci `Bruce' Adjou (adjou\_k)
        \and Antoine `Zeletochoy' Lecubin (lecubi\_a)}

\begin{document}
  \begin{titlepage}
    \begin{center}
      \includegraphics[scale=0.4]{logo_gconfs.png} \\
    \end{center}
    {\let\newpage\relax\maketitle}
  \end{titlepage}
  \tableofcontents
  \chapter{Premier Semestre}
  \section{Objectifs du Bureau pour l'année 2013-2014}
  \begin{itemize}
    \item
      Décentraliser les décisions afin que l'association ne dépende pas de son
      bureau\\ L'ancien président, Kévin Sztern, a fait part lors de l'assemblée
      générale précédente de sa décéption vis à vis de la dépendance dont fait
      preuve l'association envers son Bureau, nous avons donc pour objectif de
      réduire cette dépendance aux tâches critiques (budget, documents
      administratifs).
    \item
      Mise en place d'un système de responsables pour les conférences\\ Cet
      objectif découle du précédent : le responsable aura pour tâche de réserver
      la salle de conférence et les différentes salles de TP, de s'assurer que
      la conférence sera enregistrée.  Ceci permettra alors de libérer le bureau
      de ces différentes tâches.
    \item
      Mise en place d'un enregistrement et d'un suivi en direct des
      conférences\\ Lors de l'année précédente nous avons expérimenté
      l'enregistrement et la retransmission en direct par le biais de Mathieu
      Trentin. Celle-ci ne nous convenait pas sur plusieurs points, dont deux
      princpaux :
      \begin{itemize}
        \item
          Le temps \\ Il nous était demandé de réserver ce service au maximum
          deux semaines avant l'événement, or il ne nous est pas possible de
          planifier des conférences plus d'un mois à l'avance.
        \item
          Les enregistrements \\ Il arrivait à Mathieu Trentin de perdre les
          enregistrements et ce sans aucune possibilité de les récupérer autre
          part.
      \end{itemize}
      Il a donc été décidé d'investir dans du matériel audiovisuel afin de
      pouvoir assurer l'enregistrement et la retransmission systématique de nos
      conférences.
    \item
      Invitation de conférenciers externes\\ Jusqu'alors, les conférences
      organisées par l'assocation étaient présentées par des membres de
      l'association ou des élèves de l'EPITA. Afin de diversifier nos sujets
      nous avons décidé de faire venir des conférenciers externes à GConfs et au
      groupe IONIS.
    \item Poursuivre les partenariats conclus en 2012 - 2013
      \begin{itemize}
        \item Les partenariats ci-dessous sont destinés à être poursuivis par le bureau :
        \begin{itemize}
          \item Assistants
          \item VJN
        \end{itemize}
      \item Les partenariats suivants sont à l'étude
        \begin{itemize}
          \item Evolutek
          \item LaCity
        \end{itemize}
      \end{itemize}
  \end{itemize}
  \section{Budget prévisionnel}
  \begin{tabular}{|c|c|c|c|}
          \hline
          Ressources & Prix (en euros) & Quantité & Total\\
          \hline
          Nom de domaine & 8 & 1 & 8\\
          Hébergement & 18 & 12 & 216\\
          Ordinateur Portable & 1059.90 & 1 & 1059.90\\
          Projecteur & 743 & 1 & 743\\
          Carte d'acquisition & 514.95 & 1 & 514.95\\
          Polos & 14.4 & 40 & 576\\
          Télécommande & 30 & 1 & 30\\
          Trépied & 50 & 1 & 50\\
          Tacble de mixage & 80 & 1 & 80\\
          Câbles divers & 50 & 1 & 50\\
          Caméra & 350 & 1 & 350\\
          \hline
          Total &  & & 3676.95\\
          \hline
  \end{tabular}
  \section{Conférences réalisées durant ce semestre}
  \begin{itemize}
    \item Retour d'expérience d'un Stagiaire à Google par Pierre `delroth' Bourdon (bourdo\_p)
    \item High Frequency Trading par Julien `snooze92' Lehuen
    \item Introduction à OCAML par Philémon `philgekni' Gardet (gardet\_p)
    \item UNIX, make it simple par Paul `Dettorer' Hervot (hervot\_p), Clément `wxcafe' Hertling (hertli\_c) et Kaci `Bruce' Adjou (adjou\_k)
    \item Introduction au C\# par Raphaël `shugo' Boissel (boisse\_r)
    \item Ocaml : Under the Hood par Théophile `yroeht' Ranquet (ranque\_t)
    \item Bien démarrer son Projet par Valentin `toogy' Iovene (iovene\_v), Alexis `Horgix' Chotard (chotar\_a) et Théophile `yroeht' Ranquet (ranque\_t)
    \item Pixel Art par Thomas `bonhomme' Chassin (chassi\_t) et Pierre Guyot
    \item Graphics par Raphaël `shugo' Boissel (boisse\_r)
    \item Graphics avancée par Raphaël `shugo' Boissel (boisse\_r)
    \item 22 conférences lors de la Nuit de l'info, par : \\
      \begin{itemize}
        \item Pierre `ptishell' Surply (surply\_p)
        \item Raphaël `shugo' Boissel (boisse\_r)
        \item Paul `Dettorer' Hervot (hervot\_p)
        \item Mickaël `mogmi' Bidon (bidon\_m)
        \item Rémi `halfr' Audebert (audber\_a)
        \item Julien Déléan
        \item Kévin `Chewie' Sztern (sztern\_k)
        \item Thomas `Mackwic' Wickham
        \item Frédéric `skikepok' Lefort (lefort\_f)
        \item Jean-Luc `thiel' Bounthong (bounth\_j)
        \item Antoine `Serialk' Pietri (pietri\_a)
        \item Valentin `nitnelave' Tolmer (tolmer\_v)
        \item Antoine `Zeletochoy' Lecubin (lecubi\_a)
        \item Adrien `schischi' Schildknecht (schild\_a)
      \end{itemize}
    \item Moteurs Physiques par Sébastien Crozet (crozet\_s)
    \item GNU Bison par Valentin `nitnelave' Tolmer (tolmer\_v) et Théophile `yroeht' Ranquet (ranque\_t)
  \end{itemize}
  \section{Notation des membres}
  \includepdf{eval_s1.pdf}
  \chapter{Second Semestre}
  \section{Avancement des objectifs}
  \begin{itemize}
          \item La mise en place de responsables pour les conférences n'a pas
                  été un succès, les responsables étant pour la plupart peu
                  disponibles.\\
                  Le bureau a dû donc se charger des tâches assignées aux
                  responsables (réservation de salles, envoi d'affiches etc.)
          \item La mise en place d'un enregistrement systématique des
                  conférences a pu se faire grâce à l'achat de matériel
                  (Webcam, micros, câbles etc.), nous avons rencontré quelques
                  problèmes mais nous avons maintenant la capacité de
                  retransmettre chacune de nos conférences.\\
          \item Nous n'avons pas pu inviter de conférenciers externes à
                  l'association, cependant nous avons quelques pistes
                  (OpenSMTPd, Gouraud Shading) pour l'an prochain.\\
          \item Les partenariats avec l'association VJN et les assistants ont
                  été poursuivis, VJN était présente sur chacune de nos
                  conférences à Villejuif (UNIX, OCaml, C\#, Bien Démarrer son
                  Projet) afin de restaurer les personnes présentes à un prix
                  raisonnable.\\
                  Les assistants (ACDC, AOC) quant à eux nous ont assisté sur
                  les TP des conférences susnommées.\\
                  Nous avons eu aussi l'occasion d'organiser une conférence en
                  collaboration avec l'association LaCity concernant le Trading
                  Haute Fréquence.
  \end{itemize}
  \newpage
  \section{Budget}
  \subsection{Budget prévisionnel}
\begin{tabular}{|c|c|c|c|}
          \hline
          Ressources & Prix (en euros) & Quantité & Total\\
          \hline
          Nom de domaine & 8 & 1 & 8\\
          Hébergement & 18 & 12 & 216\\
          Ordinateur Portable & 1059.90 & 1 & 1059.90\\
          Projecteur & 743 & 1 & 743\\
          Carte d'acquisition & 514.95 & 1 & 514.95\\
          Polos & 14.4 & 40 & 576\\
          Télécommande & 30 & 1 & 30\\
          Trépied & 50 & 1 & 50\\
          Tacble de mixage & 80 & 1 & 80\\
          Câbles divers & 50 & 1 & 50\\
          Caméra & 350 & 1 & 350\\
          \hline
          Total &  & & 3676.95\\
          \hline
  \end{tabular}
  \newline
  Une partie de ces éléments a pu être achetée avec le budget restant de l'an
  dernier :
    \begin{itemize}
          \item Webcam (74,67 \euro{})
          \item Table de mixage (39,00 \euro{})
          \item Carte son Externe (29,00 \euro{})
          \item Micros (189,00 \euro{})
          \item Trépied (33,89 \euro{})
          \item Câbles divers (31,00 \euro{})
          \item Télécommande (34,79 \euro{})
  \end{itemize}
  \subsection{Budget final}
  Le budget final accordé pour l'année 2013/2014 est de 1600 \euro{}, les deux
  administrations (EPITA, EPITECH) ayant jugé l'achat d'un ordinateur portable
  et d'un projecteur comme non indispensables.\\ Nous avons décidé de sacrifier
  les polos afin de pouvoir nous permettre l'achat de la carte d'acquisition
  (qui s'est révélée plus chère que prévue) et d'un ordinateur portable moins
  cher mais pouvant subvenir à nos besoins. En effet à ce jour l'association
  dépendait de deux membres pour l'enregistrement des conférences, il nous
  était inconcevable de devoir nous appuyer constamment sur la disponibilité
  des machines de ces deux membres.\\ L'ordinateur finalement acheté est un MSI
  GP60 (699,95 \euro{}) et la carte d'acquisition est toujours la carte DVI2USB
  3.0 de chez Epiphan (697 \euro{}).
  \newline
  Le budget restant servira à la souscription à une assurance pour notre matériel
  et aux éventuelles réparations que l'ordinateur de l'assocation devra subir.
  \newpage
  \subsection{Dépenses de l'association pour l'année 2013/2014}
  \begin{tabular}{|c|c|c|c|}
          \hline
          Ressources & Prix (en euros) & Quantité & Total\\
          \hline
          Nom de domaine & 8 & 1 & 8\\
          Hébergement & 10.2 & 12 & 134.39 (frais d'installation + location)\\
          Ordinateur portable & 699.95 & 1 & 699.95\\
          Ram (4 go) & 40.75 & 1 & 40.75\\
          Câble Ethernet & 9.90 & 1 & 9.90\\
          Multiprise & 6.99 & 1 & 6.99\\
          Micros & 189 & 1 & 189\\
          Webcam & 74.67 & 1 & 74.67\\
          Table de Mixage & 39.00 & 1 & 39.00\\
          Carte son externe & 29.00 & 1 & 29.00\\
          Trépied & 33.89 & 1 & 33.89\\
          Câbles Jack/Jack Fender & 11.00 & 2 & 22.00\\
          Câbles Jack/RCA Stagg & 4.50 & 2 & 9.00\\
          Télécommande & 34.79 & 1 & 34.79\\
          Carte d'acquisition & 697.14 & 1 & 697.14\\
          \hline
          Total & & & 2028.47 \euro{}\\
          \hline
  \end{tabular}
  Budget total : 2450 \euro{}
  \newline
  Budget restant : 421.53 \euro{}
  \section{Notation des Membres}
  \includepdf{eval_s2.pdf}
  \section{Bilan annuel}
  \subsection{Conférences organisées pour l'année 2013/2014}
  \begin{itemize}
    \item Retour d'expérience d'un Stagiaire à Google par Pierre `delroth' Bourdon (bourdo\_p)
    \item High Frequency Trading par Julien `snooze92' Lehuen
    \item Introduction à OCAML par Philémon `philgekni' Gardet (gardet\_p)
    \item UNIX, make it simple par Paul `Dettorer' Hervot (hervot\_p), Clément `wxcafe' Hertling (hertli\_c) et Kaci `Bruce' Adjou (adjou\_k)
    \item Introduction au C\# par Raphaël `shugo' Boissel (boisse\_r)
    \item Ocaml : Under the Hood par Théophile `yroeht' Ranquet (ranque\_t)
    \item Bien démarrer son Projet par Valentin `toogy' Iovene (iovene\_v), Alexis `Horgix' Chotard (chotar\_a) et Théophile `yroeht' Ranquet (ranque\_t)
    \item Pixel Art par Thomas `bonhomme' Chassin (chassi\_t) et Pierre Guyot
    \item Graphics par Raphaël `shugo' Boissel (boisse\_r)
    \item Graphics avancée par Raphaël `shugo' Boissel (boisse\_r)
    \item 22 conférences lors de la Nuit de l'info, par : \\
      \begin{itemize}
        \item Pierre `ptishell' Surply (surply\_p)
        \item Raphaël `shugo' Boissel (boisse\_r)
        \item Paul `Dettorer' Hervot (hervot\_p)
        \item Mickaël `mogmi' Bidon (bidon\_m)
        \item Rémi `halfr' Audebert (audber\_a)
        \item Julien Déléan
        \item Kévin `Chewie' Sztern (sztern\_k)
        \item Thomas `Mackwic' Wickham
        \item Frédéric `skikepok' Lefort (lefort\_f)
        \item Jean-Luc `thiel' Bounthong (bounth\_j)
        \item Antoine `Serialk' Pietri (pietri\_a)
        \item Valentin `nitnelave' Tolmer (tolmer\_v)
        \item Antoine `Zeletochoy' Lecubin (lecubi\_a)
        \item Adrien `schischi' Schildknecht (schild\_a)
      \end{itemize}
    \item Moteurs Physiques par Sébastien Crozet (crozet\_s)
    \item GNU Bison par Valentin `nitnelave' Tolmer (tolmer\_v) et Théophile `yroeht' Ranquet (ranque\_t)
    \item Chiffrement et Anonymat par Nahim 'naam' Elatmani (elatma\_n) et 'genma'
    \item Networking par Clément 'wxcafe' Hertling (hertli\_c)
    \item Introduction au python par Antoine 'serialk' Pietri (pietri\_a)
    \item Don du Sang par France ADOT
    \item C++ Extravaganza par Kévin 'Chewie' Sztern (sztern\_k)
    \item Machine Learning par Quentin 'underflow' de Laroussilhe (delaro\_q)
    \item Haskell par Guillaume 'vermeille' Sanchez (sanche\_g)
    \item Administration Système à Prologin par Pierre 'delroth' Bourdon
            (bourdo\_p)
    \item IntelliJ IDEA par Valentin 'nitnelave' Tolmer (tolmer\_v)
    \item Retransmission en direct de la LSE Week le 17,18 et 19 juillet 2014
  \end{itemize}
    \subsection{État des objectifs}
    \begin{itemize}
            \item Décentraliser l'association \\
                    todo
            \item Mise en place un enregistrement et d'une retransmission en
                    direct\\
                    todo
    \end{itemize}
    \subsection{Difficultés rencontrées par l'association}
  \begin{appendices}
    \chapter{Rapports des membres}
    \includepdf{rapport_zeletochoy.pdf}
    \includepdf{rapport_seirl.pdf}
  \end{appendices}
\end{document}

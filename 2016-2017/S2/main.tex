\documentclass[12pt,a4paper]{article}
\renewcommand{\baselinestretch}{1.5}
\usepackage[utf8]{inputenc}
\usepackage[T1]{fontenc}
\usepackage[french]{babel}
\usepackage{graphicx}
\usepackage{lmodern}
\usepackage{amsmath,amsfonts,amssymb}
\usepackage{listings}
\usepackage{fancyhdr}
\usepackage{eurosym}
\usepackage{pifont}
\usepackage{avant}
\usepackage{hyperref}
\usepackage{geometry}
\usepackage{eurosym}
\geometry{top=2cm, bottom=2.5cm, left=3cm, right=3cm}

\newcommand{\HRule}{\rule{\linewidth}{0.5mm}}

\newcommand{\cmark}{\ding{51}}
\begin{document}
\begin{center}

\begingroup
\thispagestyle{empty}
\vspace*{1cm}
\HRule \\[0.8cm]
\par\normalfont\fontsize{35}{35}\sffamily\selectfont
\textbf{Rapport semestriel}\\
\HRule \\[2cm]
\includegraphics[scale=0.2]{gconfs.png}\par
{\large Association GConfs}\par
\vspace*{1cm}
{\large Année 2016-2017, Second semestre}\par
\vspace*{2cm}
\endgroup
\begingroup
\par\normalfont\fontsize{35}{16}\sffamily\selectfont
{\large Meven \emph{mevouc} Courouble (2019) - Président}\par
{\large François \emph{Hellzy} Mazeau (2019) - Vice-président}\par
{\large Alexis \emph{Horgix} Chotard (2015) - Secrétaire}\par
{\large Jean-Baptiste \emph{Zibe} Hervé (2018) - Vice-secrétaire}\par
{\large Yohann \emph{Shepard} Léon (2019) - Trésorier}\par
{\large Kaci \emph{Bruce} Adjou (2018) - Vice-trésorier}\par
\endgroup


\end{center}
\newpage

\setlength{\headheight}{15pt}
\pagestyle{fancy}
\author{Yohann LEON}
\renewcommand{\footrulewidth}{0.1 mm}
\lhead{Rapport semestriel}
\rhead{GConfs}

  \newpage
  \tableofcontents

\newpage

\section{Présentation de l'association}

GConfs est une association ayant pour objectif de faciliter et d’encourager le partage de
connaissances, aussi bien en informatique que dans d'autres domaines. Pour cela, l’association
met à disposition de conférenciers tout un panel de services afin qu’ils n’aient à se soucier que de leur sujet et de ce qui les passionne.

\section{Événements organisés}

\subsection{Présentation BDE}
\begin{itemize}
\item Responsable : Meven Courouble
\item 15 février 2017
\item Campus KB
\item Estimations de l'audimat : 200 personnes
\item Présentation des divers BDE en prévision de l'élection du nouveau BDE. Nous avons rencontré de nombreux problème technique, particulièrement au niveau de la webcam qui est tombé en panne quelques heures avant l'événement.
\end{itemize}

\subsection{Conférence LaCity}
\begin{itemize}
\item Responsable : David Lespine
\item 9 mars 2017
\item Campus KB
\item Estimations de l'audimat : 50 personnes
\item Conférence avec l'assocation LaCity en partenariat avec MargoConseil. La conférence portait sur les activités de l'entreprise et plus généralement sur les métiers de Recherche et Développement et conseil informatique dans le domaine de la finance.
\end{itemize}

\subsection{Conférence Blender}
\begin{itemize}
\item Responsable : Corentin Le Bigot
\item 16 mars 2017
\item Campus VJ
\item Estimations de l'audimat : 50 personnes
\item Au programme, une introduction, quelque techniques de base pour se lancer ainsi que des techniques (relativement) avancées pour ceux qui veulent défier Blizzard ou From Software sur le plan graphique. La conférence fut donnée par Kenan Lejosne.
\end{itemize}

\subsection{Les métiers du jeu vidéo}
\begin{itemize}
\item Responsable : François Mazeau
\item 4 mai 2017
\item Campus KB
\item Estimations de l'audimat : 20 personnes
\item Conférence à destination des élèves de cycle ingénieur en comité réduit pour des raisons de confidentialité. La conférence n'a également pas été enregistrée pour ces mêmes raisons. La conférence fut donnée par Jean-Baptiste Hervé.
\end{itemize}

\subsection{Conférence Internationale}
\begin{itemize}
\item Responsable : Raphaël Gault
\item 10 et 11 mai 2017
\item Campus VJ
\item Estimations de l'audimat : 60 personnes
\item Organisation de la présentation des destinations à l'étranger à destination des sups.
\end{itemize}

\subsection{Cryptoparty}
\begin{itemize}
\item Responsable : Fabien Tessier
\item 7 juin 2017
\item Campus KB
\item Estimations de l'audimat : 120 personnes
\item Une cryptoparty est un évènement qui a pour but de diffuser des 
connaissances et d'échanger sur les concepts de vie privée et sur
la sécurité informatique en général.
En général, les cryptoparties ont une orientation pratique, plus que théorique :
le but n'est pas ici d'apprendre comment fonctionne TOR ou PGP, mais
comment utilliser ces outils pour améliorer sa vie numérique et mieux protéger sa vie privée.
Les ateliers qui ont été organisés furent divisés en plusieurs thèmes comme le mobile, les cryptomonnaies, les emails, le messaging et les réseaux et systèmes.
L'ambiance fut conviviale, avec un véritable échange entre les personnes
présentes, organisateurs comme visiteurs. Le but était de répondre aux questions
de ces derniers, et de les aider a mieux envisager les enjeux de leur vie privée et
de la sécurité qui les touchent, et non de leur faire un cours général et générique. C'était aussi l'occasion pour GConfs d'essayer un nouveau format d'événements, différents des confs "à sens unique" habituelles d'un speaker à son audience.
\end{itemize}

\section{Informations importantes}

\subsection{Conférence EuroBSDCon}

Il y a quelques mois, nous avons été contactés par Antoine Jacoutot, l'un des organisateurs de l'EuroBSDcon. L'EuroBSDcon est une série de conférences annuelle sur les sytèmes d'exploitation libres de type BSD (FreeBSD, NetBSD, OpenBSD, Darwin...). La conférence aura lieu cette année à l'Espace S$^\text{t}$ Martin dans le troisième arrondissement de Paris les 23 et 24 septembre 2017. Cet événement rassemble environ 200 développeurs venant de toute l'Europe.

Cette demande fait suite à la participation de M. Jacoutot à la LSE Week 2016, qui a pu constater l'efficacité de GConfs lors de ce genre d'événement. L'aide de M. Espie nous a également été très précieuse.  Il nous a été demandé de re-transmettre simultanément trois scènes, accueillant chacune une conférence.

Nous avons d'ores et déjà commencé les préparatifs en prévision de cette conférence. En effet, EPITA et GConfs sont désormais officiellement sponsors de l'événement et leur logo est présent sur le site web de l'événement à l'adresse \url{https://2017.eurobsdcon.org/sponsors/}. 

De plus, une première visite de l'espace d'exposition en compagnie de M. Jacoutot est prévu le 10 juillet afin de faire des repérages sur site. Grâce au budget spécial qui nous a été accordé lors du dernier CIA, nous avons commencé à faire des achats, comme une nouvelle webcam et un nouveau jeu de micros. Le reste du matériel manquand sera acheté pendant l'été suite à la visite des locaux.

\subsection{Local}

Suite aux travaux, certaines associations ont temporairement déménagées dans le sous-sol Pasteur. GConfs a cependant trouvé un arrangement avec le LSE pour que notre matériel soit conservé dans leur laboratoire. Cependant, cette solution n'est que temporaire et ne pas pouvoir bénéficier d'un local s'est avéré quelques peu contraignant lors de l'organisation d'événement ou de réunions.

\subsection{Assemblée Générale}

Une assemblée générale a eu lieu le 28 juin pour élire un nouveau bureau. Ce nouveau bureau est est le suivant :

\begin{itemize}
\item Président : Meven \emph{mevouc} Courouble (2019)
\item Vice-président : Victor \emph{multun} Collod (2020)
\item Trésorier : Yohann \emph{shepard} Léon (2019)
\item Vice-trésorier : Paul \emph{Eazhi} Khuat Duy (2021)
\item Secrétaire : Quentin \emph{Sunbro} Barbarat (2021)
\item Vice-secrétaire : Fabien \emph{Satan} Tessier (2019)
\end{itemize}

Le nouveau bureau a été monté en binôme par Yohann \emph{shepard} Léon, qui était
trésorier de l'association cette année, et Meven \emph{mevouc} Courouble qui était président de l'assocation cette année. La dynamique de GConfs ayant été plutôt
bonne cette année, nous avons donc décidé de  conserver la gestion de l'asso et de monter un bureau.
Tous les membres du bureau proposé n'ont jamais eu un rôle au bureau
précédemment, sauf Yohann et Meven. Les nouveaux membres du bureau sont des membres
qui ont été remarqué comme étant particulièrement actifs au sein de GConfs cette année. Ces nouveaux membres sont aussi volontairement issus pour la plupart de promos plus jeunes que le précédent bureau, afin d'encourager le passage du relai pour les années à venir.

Le programme de cette liste est de poursuivre ce qui a été fait en
2016/2017, réessayer les types d'événement qui ont bien marché
(nouvelle conférence de présentation des associations en septembre,
soirées de conférence à caractère non-informatique à l'image de celle organisée lors des 10 ans de GConfs, conférences de type Meetup comme la Cryptoparty, partenariat conséquent comme l'EuroBSDcon), et essayer de nouvelles choses, comme l'organisation d'une GameJam mêlée à des conférences.

\subsection{Conférences en province}

Au sein de l'assocation, nous avons également abordé l'idée de réaliser une conférence de type 'Bien Démarrer Son Projet' à destination des étudiants en province. Bien que nous soyons encore en discussion sur sa viabilité, nous sommes ouvert à une collaboration avec l'administration pour la concrétiser.

\subsection{Incident de répartition des budget}

Suite à une erreur, nous nous sommes vu attribuer le budget du BDE Epitech sur notre compte associatif pour un montant total de 5618 euros. Après une prise de contact, nous avons restituée la somme dûe au BDE Epitech.

\section{Autres événements}
\subsection{Conférence LSE Week}

GConfs a également répondu présent cette année encore pour la LSE Week qui se déroulera le 14 et 15 juillet à EPITA.

\subsection{JDMI}

Nous avons pu organiser plusieurs ateliers de programmation à destination des lycéens pendant les JDMI.

\subsection{Concours GPGE}

GConfs a répondu présent lorsqu'elle fut conviée pour assister à l'accueil des candidats au concours CPGE.

\subsection{JPO}

L'association fut présente à chaque JPO où elle fut conviée.

\end{document}

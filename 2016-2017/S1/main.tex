\documentclass[12pt,a4paper]{article}
\renewcommand{\baselinestretch}{1.5}
\usepackage[utf8]{inputenc}
\usepackage[T1]{fontenc}
\usepackage[french]{babel}
\usepackage{graphicx}
\usepackage{lmodern}
\usepackage{amsmath,amsfonts,amssymb}
\usepackage{listings}
\usepackage{fancyhdr}
\usepackage{eurosym}
\usepackage{pifont}
\usepackage{avant}
\usepackage{hyperref}
\usepackage{geometry}
\usepackage{eurosym}
\geometry{top=2cm, bottom=2.5cm, left=3cm, right=3cm}

\newcommand{\HRule}{\rule{\linewidth}{0.5mm}}

\newcommand{\cmark}{\ding{51}}
\begin{document}
\begin{center}

\begingroup
\thispagestyle{empty}
\vspace*{1cm}
\HRule \\[0.8cm]
\par\normalfont\fontsize{35}{35}\sffamily\selectfont
\textbf{Rapport semestriel}\\
\HRule \\[2cm]
\includegraphics[scale=0.2]{gconfs_logo.png}\par
{\large Association GConfs}\par
\vspace*{1cm}
{\large Année 2016-2017, Premier semestre}\par
\vspace*{2cm}
\endgroup
\begingroup
\par\normalfont\fontsize{35}{16}\sffamily\selectfont
{\large Meven \emph{mevouc} Courouble - Président}\par
{\large François \emph{Hellzy} Mazeau - Vice-président}\par
{\large Alexis \emph{Horgix} Chotard - Secrétaire}\par
{\large Jean-Baptiste \emph{Zibe} Hervé - Vice-secrétaire}\par
{\large Yohann \emph{Shepard} Léon - Trésorier}\par
{\large Kaci \emph{Bruce} Adjou - Vice-trésorier}\par
\endgroup


\end{center}
\newpage

\setlength{\headheight}{15pt}
\pagestyle{fancy}
\author{Yohann LEON}
\renewcommand{\footrulewidth}{0.1 mm}
\lhead{Rapport semestriel}
\rhead{GConfs}

  \newpage
  \tableofcontents

\newpage

\section{Présentation de l'association}

GConfs est une association ayant pour objectif de faciliter et d’encourager le partage de
connaissances, aussi bien en informatique que dans d'autres domaines. Pour cela, l’association
met à disposition de conférenciers tout un panel de services afin qu’ils n’aient à se soucier que de leur sujet et de ce qui les passionne. \\
L’association organise également des conférences, principalement techniques ou
sur le monde du travail et ayant pour but d’enrichir le cursus par des thèmes s’y intégrant ou s’en éloignant.

\section{Événements organisés}
\subsection{Introduction à Linux}
\begin{itemize}
\item Responsable : Yohann Léon
\item 14 septembre 2016
\item Campus KB
\item Conférence d'introduction à Linux pour les élèves en SPÉ suivie d'un TP.
\end{itemize}

\subsection{Présentation des associations}
\begin{itemize}
\item Responsable : Meven Courouble
\item 16 septembre 2016
\item Campus VJ
\item Nuit de présentation des associations d'EPITA, EPITECH et SUBIOTECH à destination des SUP, SPÉ et ING1. Les associations Unisson, Evolutek, Sup'Bio Dance, Synergie, Antre, VJN, Prologin, Lateb, Ephemère, Epinavet, Stack, Au'r Kestra, Epirev et EpiOeno étaient présentes.
\end{itemize}

\subsection{Conférence Git}
\begin{itemize}
\item Responsable : Thomas Michelot
\item 07 novembre 2016
\item Campus VJ
\item Introduction au système de versionning Git.
\end{itemize}

\subsection{Bien négocier son salaire}
\begin{itemize}
\item Responsable : Jean-Baptiste Hervé
\item 19 octobre 2016
\item Campus KB
\item Conférence à destination des élèves de cycle ingénieur. La conférence n'a pas été enregistrée pour des raisons de confidentialité. La conférence a été donnée par Nassim Eddequiouaq (EPITA 2016).
\end{itemize}

\subsection{Conférence des 10 ans de GConfs}
\begin{itemize}
\item Responsable : Meven Courouble
\item 9 et 10 novembre 2016
\item Campus KB
\item Deux soirées de conférences festives pour commémorer les 10 ans de notre association. 22 conférences, dont 2 externes (Hana Gauër, Clément Hertling), 2 anciens (Antoine Lecubin, Alexandre Abraham) et 2 professeurs de l'école (Didier Verna, Olivier Ricou).
\end{itemize}

\subsection{Conférence C\#}
\begin{itemize}
\item Responsable : Florent Bordignon
\item 18 novembre 2016
\item Campus VJ
\item Introduction au langage C\# à destination des étudiants de SUP. La conférence a été suivie d'un TP, durant lequel plusieurs ING1, dont beaucoup d'ACDC sont venus aider.
\end{itemize}

\subsection{Nuit de l'Info}
\begin{itemize}
\item Responsable : Kenan Lejosne
\item 01 décembre 2016
\item Campus KB
\item Nuit de conférences à l'occasion de La Nuit de l'Info. 9 conférences et un TP par Julien 'Snooze' Lehuen (EPITA 2013).
\end{itemize}

\subsection{Bien démarrer son projet}
\begin{itemize}
\item Responsable : François Mazeau
\item 19 janvier 2017
\item Campus VJ
\item Conférence d'introduction à Unity, conférence protips et Git à destination des étudiants en SUP. La conférence a été suivie d'un TP, durant lequel plusieurs ING1, dont beaucoup d'ACDC sont venus aider.
\end{itemize}

\newpage

\section{Événements particuliers}

\subsection{Présentation des projets de SUP}

\begin{itemize}
\item Responsable : Alexandre Kirszenberg
\end{itemize}

Comme à l'accoutumée, GConfs a répondu présent pour organiser en étroite collaboration avec la pédagogie du campus VJ la présentation des projets de jeux vidéos de SPÉ aux étudiants en SUP.

\subsection{JDMI}

Nous avons pu organiser plusieurs ateliers de programmation à destination des lycéens pendant les JDMI.

\subsection{Assistants}

À l'occasion de certaines conférences, GConfs organise des TP (exercices de programmation) pour que les spectateurs puissent appliquer ce qu'ils viennent d'apprendre. Les assistants ACDC ont répondu présent lors de plusieurs conférences pour aider à leur organisation.

\textit{In theory, theory and practice are the same. In practice, they're not.}

\subsection{JPO}

L'association a répondu présente à plusieurs JPO de l'EPITA.

\end{document}
